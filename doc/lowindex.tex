\documentclass[12pt]{article}
\usepackage[width=6.5in,height=10in]{geometry}
\usepackage{tgpagella}
\usepackage{amsthm}
\usepackage{amsmath}
\usepackage{amsrefs}

\parindent 0pt 
\parskip 1ex
\newtheorem{theorem}{Theorem}
\theoremstyle{definition}
\newtheorem{definition}{Definition}
\renewcommand{\tilde}{\widetilde}
\begin{document}
\normalfont
\title{LOW~INDEX, Revisited and Implemented}
\author{Marc Culler, Nathan Dunfield, and Matthias G\"orner}
\maketitle
\begin{abstract}\noindent
  This note provides a simple description of Charles Sims' LOW~INDEX
  algorithm in terms of elementary covering space theory.  It
  accompanies a working open source implementation of the algorithm.
\end{abstract}
\parindent 0pt 
\parskip 1ex 
\section{Introduction}

In his $1994$ memoir \cite[Chapter 5]{Sims} Charles Sims presented a pseudo-code
description of an algorithm for enumerating all conjugacy classes of subgroups
of a finitely presented group $G$ with index at most a specified integer $N$.
The input to the algorithm is a finite presentation of a group and the output is
a system of distinct representatives for conjugacy classes of subgroups of index
at most $N$.  He named the algorithm LOW\_INDEX.  The algorithm has been
implemented in a number of computer algebra systems, including the proprietary
MAGMA system and the open source GAP system.  It is considered to be the most
efficient general purpose algorithm for this task.

The description in Sims' book is lengthy and technical, requiring several
chapters of preparation before the ideas of the algorithm can be described.  In
this note we will present a simplified description of the algorithm which is
accessible to anyone who is familiar with the basic theory of covering spaces of
graphs as presented in the classic undergraduate text book written by William
Massey \cite{Massey}.

Breaking with a long tradition in the publication of mathematical algorithms,
this note is being published as part of a freely distributable open source
implementation of the algorithm which also happens to be faster than any other
implementation known to the authors.  The implementation is configured as a
package for the Python programming language, based on a Python extension module
written in C++ and wrapped using pybind-11.  The package uses parallel
computation for optimal speed, but it is actually faster than the
implementations mentioned above even when restricted to a single CPU.

\section{Covering Spaces and Conjugacy Classes}\label{covers}

We review the two theorems about covering spaces that will be used in this note.
The first is what Massey calls the existence theorem for covering spaces.  It
says that every conjugacy class of subgroups of the fundamental group of a
connected CW complex contains the image of the fundamental group of a
covering space under the homomorphishm induced by the covering map.

Since the topological spaces that we consider are all CW complexes, which are
locally arcwise connected and semi-locally simply-connected, we paraphrase the
statement to fit our situation.  Note that $(X,x)$ is used to denote a
topological space $X$ with a chosen base point $x$.  A $0$-cell in a CW-complex
will be called a {\it vertex} and a $1$-cell will be called an {\it edge}.

\newtheorem*{thm}{Theorem}
\begin{thm}[\cite{Massey} Theorem 10.2]
  Let $X$ be a connected CW complex and $x$ a vertex of $X$.  Then given any
  conjugacy class of subgroups of $\pi_1(X, x)$ there exists a covering map
  $p:(\tilde X, \tilde x) \to (X, x)$ such that $p_*(\pi_1(\tilde X, \tilde x))$
  belongs to the given conjugacy class.
\end{thm}

Combining this with the next result, also paraphrased from \cite{Massey}, we see that
each conjugacy class in the fundamental group of a connected CW complex can be
associated to a single covering space, and that every subgroup within the class
is obtained as the induced image of the fundamental group of that covering space
with respect to some choice of base point.

\begin{thm}[\cite{Massey} Theorem 4.2]
  Let $X$ be a connected CW complex and $p:(\tilde X, \tilde x )\to(X, x)$ a
  covering map, where $x$ is a vertex of $X$. Then the set of subgroups of
  $\pi_1(X, x)$ which have the form $p_*(\tilde X, \tilde x)$, as $\tilde x$
  ranges over all vertices of $\tilde X$ which are contained in $p^{-1}(x)$, is
  exactly a conjugacy class in $\pi_1(X, x)$.
  \end{thm}

  We are interested in enumerating all conjugacy classes of finite index
  subgroups of a given finitely presented group $G$ which have index at most
  $N$.  For a covering map $p:(\tilde X, \tilde x)\to (X, x)$, the index of
  $p_*(\pi_1(\tilde X, \tilde x))$ in $\pi_1(X, x)$ is equal to the degree of
  the covering.  So our enumeration task can be divided into the following
  subtasks:

  \begin{itemize}
  \item Construct a CW-complex $X$ with base vertex $x$ such that
    $\pi_1(X, x)$ is isomorphic to $G$.
   \item Describe a way of choosing a canonical base vertex in any
     covering of $X$ (meaning that each covering has exactly one caononical base vertex).
   \item Enumerate all covering maps $(\tilde X, \tilde x)\to (X,
     x)$ of degree at most $N$ such that $\tilde x$ is the canonical base
     vertex of $\tilde X$
   \end{itemize}

   \section{Cayley Complexes}

   There is a standard construction of a $2$-dimensional CW-complex whose
   fundamental group is isomorphic to a given finitely presented group.  The
   complex is called the {\it Cayley complex} of the presentation, or the {\it
     presentation complex}.  A description of the construction can be found in
   Allen Hatcher's textbook on Algebraic Topology \cite[Section 1.3]{Hatcher}.
   We will review it here, paying attention to some additional combinatorial
   structure which is inherent in the construction and which will be used in the
   data structures that appear in our implemention of Sims' algorithm.

   Suppose we are given a presentation with $r$ generators and $k$ relators
   $|x_1, \ldots, x_r : w_1, \ldots, w_k|$.  The Cayley complex $C$ of this
   presentation will have one $0$-cell $v$, $r$ edges $e_1, \ldots, e_r$ and
   $k$ $2$-cells $d_1, \ldots, d_k$. The attaching map of the edge $e_i$
   sends both boundary points of the cell to the vertex.  To describe the
   attaching map of a $2$-cell it is useful to fix a homeomorphism
   $h_i : [0,1] \to e_i$ for each $i = 1, \ldots, r$.  By composing $h_i$ with
   the inclusion map from $e_i$ to the $1$-skeleton $C^{(1)}$ of $C$ we obtain a
   closed loop $\gamma_i$, based at $v$, in $C^{(1)}$.  The homotopy classes of
   these loops form a basis of the free group $\pi_1(C^{(1)}, v)$.  If the
   relator $w_j$ has length $n$ and is expressed as the word
   $x_{i_1}\cdots x_{i_n}$ then the attaching map for the $2$-cell is the path
   $\gamma_{i_1}\ast\cdots\ast\gamma_{i_n}$, where $\ast$ denotes the path
   composition operator.  It is straightforward \cite[Corollary 1.28]{Hatcher}
   to use the Seifert - Van Kampen Theorem to show that the fundamental group of
   $C$ is isomorphic to the group given by our presentation, where the
   isomorphims sends the based homotopy class of $\gamma_i$ to the generator $x_i$.

   This description of $C$ includes some additional structure which will be
   useful to give a purely combinatorial description of $C$ that can be used in
   an implementation of Sims' algorithm.  First, each edge $e_i$ of $C$ is
   oriented, with the orientation being induced by the homeomorphism $h_i$.
   This gives the $1$-skeletion of $C$ the structure of a directed graph.
   Second, each edge $e_i$ has a label, namely the positive integer
   $i\in\{1, \ldots, r\}$, giving the $1$-skeleton the structure of a labeled
   directed graph.  Since the paths $h_i:([0,1], 0)\to (C, v)$ can be lifted to
   any covering map $(\tilde C, \tilde v) \to (C, v)$, these orientations and
   labels can be pulled back under a covering map to induce the structure of a
   labeled directed graph on the $1$-skeleton of any covering space of $C$.
   
   \subsection{Restricting to the $1$-skeleton}

   Here we make the observation that it is not necessary to work with
   $2$-dimensional CW complexes in our setting. If $(C, v)$ is the Cayley
   complex of a finite group presentation with base point at the unique vertex
   $v$ of $C$, then the $1$-skeleton of any covering space of $(C, v)$ is a
   covering space of the $1$-skeleton $(C^{(1)}, v)$ of $C$ under the
   restriction of the covering map.  Moreover it is a covering space with the
   property:
   \begin{itemize}
   \item[$(\ast)$]  The attaching map of each $2$-cell, viewed as a loop in the
     $1$-skeleton as above, lifts to every vertex in the fiber over $v$.
   \end{itemize}
   Conversely, a covering space of $(C^{(1)}, v)$ with the property $(\ast)$
   arises as the $1$-skeleton of a covering space of $(C, v)$ constructed by
   attaching a $2$-cell covering $d$ along every lift of the attaching map of $d$
   for each $2$-cell $d$.

   So, the based finite covering spaces of $(C, v)$ are in one-to-one
   correspondence with the finite covering spaces of $(C^{(1)}, v)$ which have
   property $(\ast)$.  The latter can be simply characterized as directed graphs
   with edges labeled by integers in $\{1, \ldots r\}$ such that each vertex is
   incident to exactly one incoming edge and one outgoing edge with each label.

   \subsection{Slots}

   Typically one views a CW complex as a quotient space constructed from a
   disjoint union of compact cells by attaching cells of dimension $n$ to the
   $(n-1)$-skeleton via their attaching maps.  For our application, however, it
   will be convenient to work with a decomposition into open subsets (as also is
   the case when applying the Seifert - Van Kampen Theorem).  We only need to
   consider connected $1$-dimensional CW complexes, so we limit the discussion
   to that case.

   Given a finite $1$-dimensional CW complex $C$ with vertices $v_1, \ldots,
   v_d$ and edges $e_1, \ldots, e_r$ choose a point $p_i$ in the interior of
   each edge $e_e$ and consider the subspace $V = C - \{p_i, i=1, \ldots, r$.
   The space $V$ is a union of contractible open sets, each containing exactly
   one vertex.  Let $N_i$ denote the component of $V$ containing the vertex
   $V_i$.  The components of $N_i - v_i$ are called the {\it slots} of the
   vertex $v_i$.  A slot is homeomorphic to the open interval $(0, 1)$.
   Each edge $e_i$ of $C$ contains exactly two slots.  We may construct $C$
   as a quotient space of the disjoint union of the sets $N_i$ and the interiors
   of the edges $e_i$ by identifying each slot with one component of the
   interior of $e_i - p_i$.

   In our setting each edge is oriented and labelled, where the label of
   $e_i$ is the integer $i$.  The edge $e_i$ containes two slots.  One of
   these is called the {\it initial} slot and the other is called the {\it
     terminal} slot, where the names are chosen so that the points of the
   initial slot precede those of the terminal slot with respect to the ordering
   of $e_i$ given by its orientation.  The initial slot in $e_i$ is labeled with
   the pair of positive integers $(m, i)$ where $N_m$ is the open set as above
   which contains the slot.  The terminal slot in $e_i$ is labeled by the pair
   $(n, -i)$, where $N_n$ contains the slot.  Thus we may say that the
   edge $e_i$ is the outgoing edge at vertex $v_m$ in the slot labeled $i$
   and also the incoming edge at the vertex $v_n$ in the slot labeled $-i$.

   The labels on the slots are unique and we also impose an ordering on them
   as follows.  If our CW complex has $d$ vertices and $r$ edges then
   $$(1, 1) < (1, -1) < \cdots < (1, r) < (1, -r) < \cdots (d, 1) < (d, -1) <
   \cdots (d, r) < (d, -r) .$$

   This corresponds, in a free group of rank $r$,  to ordering the basis
   elements and their inverses as:
   $$x_1 < x_1^{-1} < \cdots x_r < x_r^{-1}.$$
   This ordering of basis elements has the nice feature that it is preserved
   under the usual embedding of a free group of rank $r$ into a free group of
   rank $r+1$.

   \subsection{Covering Subgraphs}
   The most prevalent object in our implementation of Sims' algorithm is called
   a {\it covering subgraph}.  We give the definition here.

   Consider a $1$-dimensional CW complex $B$ with one vertex $v$ and $r$ edges
   which are oriented and labeled by the integers $1, \ldots, r$.  For each edge
   $e_i$, $i = 1, \ldots, r$, choose a point $p_i$ in the interior of $e_i$.  Let
   $N = N_v$ be the contractible open neighborhood of $v$ given by
   $N = B - \{p_1, \ldots, p_r\}$.

   A {\it covering subgraph} $C$ of $B$ with degree $d$ is a connecte
   topological space constructed from a degree $d$ covering map
   $p:\tilde B \to B$ where the edges of $\tilde B$ are oriented and labeled by
   lifting the labels and orientations from $B$.  The space $C$ is given as
   $\tilde B - P$ where $P$ is some subset of $p^{-1}(\{p_1, \ldots, p_r\})$.
   The vertices of $C$ are labeled by integers in $\{1, \ldots, d\}$.  Two
   covering subgraphs are isomorphic if there is a homeomorphism between them
   which preserverses labels of vertices and edges and preserves the orientations
   of the edges.

   Abstractly, a covering subgraph can be constructed from a disjoint union of
   $d$ copies of $N$, named $N_1, \ldots, N_d$, by adding at most $rd$ oriented
   open edges.  Let us denote the vertex in $N_i$ as $v_i$.

   The construction assumes that each edge has been labeled with an integer in
   the set $\{1, \ldots, r\}$ in such a way that there are at most $d$ edges
   with each label.  An open edge labeled $i$ is added by identifying an initial
   interval of the edge with the slot $(m, i)$ (i.e. the slot in $N_m$ labeled
   $i$) and a terminal interval of the edge edge with a slot labeled $(n,-i)$.
   The initial and terminal intervals are obtained as the components of the
   complement of a point in the open edge.  After making these identifications
   the covering subgraph will have an edge labeled $i$ from the vertex $v_m$ to
   the vertex $v_n$.  It is required, of course, that each slot be used by at
   most one edge and that the resulting space be connected.

   It is immediate from the first description that a covering subgraph $C$ of
   $B$ with degree $d$ admits a local homeomorphism to $B$, such that the local
   homeomorphims is a covering map of degree $d$ if and only if $C$ has $rd$
   edges (i.e. none of the points $p_i$ have been removed).  We define a
   covering subgraph to be {\it complete} if it has $rd$ edges.  We say that a
   slot of the covering subgraph $C$ is {\it empty} if there is no edge containing
   it, i.e. if a point has been removed from the edge in $\tilde B$ containing
   it.

   In order to implement Sims' algorithm we need a completely combinatorial
   description of a covering subgraph which, as it turns out, is readily available.
   A covering subgraph is completely described by specifying the initial and
   terminal vertices of each of its edges.  This information can be encoded in
   the {\it outgoing matrix} of the covering subgraph, which is the $r\times d$
   matrix $\mathop{Out(C)} = (a_{ij})$ defined as follows:
   \[ a_{ij} =
     \begin{cases}
       k & \text{if there is an edge of $C$ labeled $j$ from $v_i$ to $v_k$}\\
       0 & \text{otherwise}\\
     \end{cases}
   \]
 The analagous $r\times d$ {\it incoming matrix} $\mathop{In(C)} =
 (a_{ij})$ is defined by
   \[ a_{ij} =
     \begin{cases}
       k & \text{if there is an edge of $C$ labeled $j$ from $v_k$ to $v_i$}\\
       0 & \text{otherwise}\\
     \end{cases}
   \]
   Either of these matrices describes the covering subgraph, and each can be
   derived from the other.  However, our implementation of Sims' algorithm
   actually records both of them, in order to make the process of lifting
   relators more efficient.  This pair of matrices is equivalent to the
   ``transition matrix'' that appears in Sims' description of his algorithm, and
   to the ``coset table'' that appears in some other descriptions of his
   algorithm.
       
   \subsection{The Canonical Base Point}
   As was mentioned in Section \ref{covers} our scheme for enumerating all
   conjugacy class of subgroups of index $N$ requires that there be a canonical
   choice of base point for each finite covering space $\tilde B$ of the CW
   complex $B$ with $1$ vertex and $r$ directed labeled edges
   $e_1, \ldots, e_r$.  In this subsection we will describe how to characterize
   the canonical base point for such a cover.

   So far, each degree $d$ covering space of $B$ that we have discussed has had
   labeled vertices with labels $1, \ldots, d$.  But these labels were
   arbitrary.  The first step in describing the canonical base point is to
   observe that each choice of a base vertex $v_1$ in $\tilde B$ determines a
   natural labeling of all of the vertices of $\tilde B$ such that $v_1$ has
   label $1$.  This natural labeling is produced by the following inductive
   procedure which also generates a sequence of covering subgraphs
   $C_1, C_2, \ldots, C_{rd} = \tilde B$ such that $C_{i+1}$ is constructed from
   $C_i$ by adding a single edge and such that each vertex of $C_i$ has the same
   label $C_{i+1}$ as it had in $C_i$. Recall that we have defined an ordering
   of the slots of a covering subgraph with labeled vertices.
   \begin{itemize}
   \item $C_1$ consists of the vertex $v_1$ and has no edges.
   \item For $j > 1$ let the first empty slot $s$ in $C_j$ have label $(i, l)$.
     If $l > 0$ then construct $C_{i+1}$ from $C_i$ by adding the edge $e$
     of $\tilde B$ whose initial slot is $s$.  If $l < 0$ then construct
     $C_{i+1}$ from $C_i$ by adding the edge $e$ of $\tilde B$ whose terminal
     slot is $s$.  If there is a vertex of $C_{i+1}$ which is not a vertex of $C_i$ then
     assign it the label $1 + V$ where $V$ is the number of vertices of $C_i$.
   \end{itemize}

   Next suppose that we are given a finite cover $\tilde B$ of $B$ with an
   arbitrary labeling of its vertices.  Given a slot $s$ with label $(n, l)$ let
   $E_s$ denote the edge of $\tilde B$ containing $s$.  If $l > 0$ then $E_s$ is an
   outgoing edge at the vertex with label $n$ and we let $V_s$ denote the
   label of the terminal vertex of $E_s$.  If $l < 0$ then $E_s$ is an incoming
   edge at the vertex labeled $n$ and we let $V_s$ denote the label of the
   initial vertex of $E_s$.  We may then define the {\it complexity} of our
   vertex labeling to be the tuple of positive numbers $(V_s)$ indexed by the
   (ordered) slots in $\tilde B$.  The complexities are ordered
   lexicographically.

   \begin{definition}
     A vertex $v$ of a finite covering space $\tilde B$ is {\it canonical} if
     $v$ has minimal complexity among all vertices of $\tilde B$.
   \end{definition}

   There are two observations to make here.  First, the complexity is an
   invariant of the isomorphism class of the based covering.  That is, if two
   vertices $v$ and $w$ generate the same labeling then there is an isomorphism
   of based coverings from $(\tilde B, v)$ to $(\tilde B, w)$.  Second, the
   complexity makes sense for a covering subgraph which is not complete,
   if we extend the definition by defining $V_s$ to be $0$ if the slot $s$ is empty.

   The last observation plays a key r\^ole in the algorithm.  It means that it
   is possible to determine that a vertex of a covering subgraph could not
   possibly be canonical in any complete covering subgraph obtained by adding
   additional edges.  This conclusion follows when the (extended) complexity of
   the vertex is not minimal among the complexities of all vertices of the
   covering subgraph.
   
\section{Description of the Algorithm}
   \subsection{Permutation Representations}
   \subsection{The Sims Tree}

\section{Performance}

\begin{bibdiv}
\begin{biblist}
\bib{Sims}{book}{
   author={Sims, Charles C.},
   title={Computation with finitely presented groups},
   series={Encyclopedia of Mathematics and its Applications},
   volume={48},
   publisher={Cambridge University Press, Cambridge},
   date={1994},
   pages={xiii+604},
   isbn={0-521-43213-8},
   review={\MR{1267733}},
   doi={10.1017/CBO9780511574702},
 }
\bib{Massey}{book}{
   author={Massey, William S.},
   title={Algebraic topology: an introduction},
   series={Graduate Texts in Mathematics, Vol. 56},
   note={Reprint of the 1967 edition},
   publisher={Springer-Verlag, New York-Heidelberg},
   date={1977},
   pages={xxi+261 pp. ISBN 0-387-90271-6},
   review={\MR{0448331}},
 }
 \bib{Hatcher}{book}{
   author={Hatcher, Allen},
   title={Algebraic topology},
   publisher={Cambridge University Press, Cambridge},
   date={2002},
   pages={xii+544},
   isbn={0-521-79160-X},
   isbn={0-521-79540-0},
   review={\MR{1867354}},
}
 \end{biblist}
\end{bibdiv}
\end{document}
